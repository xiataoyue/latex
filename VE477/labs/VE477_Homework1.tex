\documentclass[12pt, a4paper]{article}
\usepackage[UTF8]{ctex}
\usepackage{enumerate}
\usepackage{amsmath}
\usepackage{amssymb}
\usepackage{blkarray}
\usepackage{geometry}
\usepackage{float}
\usepackage{graphicx}
\usepackage[linesnumbered, ruled]{algorithm2e}
\usepackage{listings}
\usepackage{xcolor}

\lstset{
    columns=fixed,
    numberstyle=\tiny\color{gray},
    frame=none,
    backgroundcolor=\color[RGB]{245,245,244},
    keywordstyle=\color[RGB]{40,40,255},
    numberstyle=\footnotesize\color{darkgray},
    commentstyle=\it\color[RGB]{0,96,96},
    stringstyle=\rmfamily\slshape\color[RGB]{128,0,0},
    showstringspaces=false,
    language=c,
}

\geometry{left = 2.0cm, right = 2.0cm}


\title{
    VE477 Introduction to Algorithms\\ 
    Lab 1}
\author{Taoyue Xia, 518370910087}
\date{2021/09/28}

\begin{document}
\maketitle

\section{C Programming}

This part of work has been uploaded to JOJ.

\section{Functional Programming}
\begin{enumerate}
    \item Imperative languages consist commands for computers to operate, they mains focus on describing how a program operates. 
          For example, \textbf{C, C++, and Java}, etc. are imperative languages.

          Object-oriented languages use objects as the basic structure unit. Attributes and methods are defined inside the object, 
          and different objects can interact with each other. Classes are the most common realization of an object. 
          Common examples include \textbf{Java, C++, Python, Javascript} and so on.

    \item Functional programming is a programming paradigm where programs are constructed by applying and composing functions. 
          It is a declarative programming paradigm in which function definitions are trees of expressions that map values to other values, 
          rather than a sequence of imperative statements which update the running state of the program.
          Common examples include \textbf{OCaml, Haskell, Common Lisp} and so on.

    \item It means that functions can be bound to names, passed as arguments and returned from other functions, just as any data type can.
    
    \item To prove that integration defines a linear map, we should prove its additivity and homogeneity. 
          First, the additivity is proved below:
          $$\int (f(x) + g(x))\, dx = \int f(x)\, dx + \int g(x)\, dx$$
          Then, the homogeneity is shown below, with a constant $c$:
          $$\int cf(x)\, dx = c\int f(x)\, dx$$
          Therefore, with the above two properties proved, we are sure that integrity is a linear map.

          For integrity a linear map, we can input the functions $f$ and $g$ as parameters of integrity, which proves to be a higher order function.
    
    \item If a variable is immutable, it means that once the variable is initialized, its value cannot be changed.
    
    \item The advantages of immutable variables:
          \begin{itemize}
              \item No need to make copies when passed into functions, thus accelerating running time and saving memory use.
              \item Since it cannot be modified, it is easy to predict its state.
              \item It becomes easier to test the program.
          \end{itemize}

          The disadvantages of immuatable variables:
          \begin{itemize}
              \item Since it cannot be modified, you need to create a new variable and copy its value. 
                    Therefore, if the variable is large, it may consume a lot of time and space.
          \end{itemize}
    
          \item A pure function is a function where the return value is only determined by its input values, without observable side effects. 
                With the same input, it will return the same result.

\end{enumerate}

\section{Getting started with OCaml}
\begin{enumerate}
    \item REPL is the acronym of read-eval-print loop, is a simple interactive computer programming environment that takes single user inputs, 
          executes them, and returns the result to the user.

    \item OCaml us a compiled language.
    
    \item C is faster because C enables programmers to do lower-level stuff which optimizes the compiler.
    
    \item The \textbf{let} keyword is used for binding names with values, in the way of definitions.
    
    \item The basic types are integers, float-point numbers, booleans, characters and immutable character strings.
    
    \item Type inference is that the language can automatically detect the type of an expression.
    
    \item It means that a function takes two integers as inputs and returns a float-point number as output.
    
    \item NULL reference problem means that you want to use a reference which is not initialized or is null.
    
          The keyword \textbf{option} makes it possible to assign a variable as \textbf{None} or with values.
    
    \item Statically typed languages define the type of variables at compile time. 
          Dynamically typed languages define variables of which the type is checked during runtime. Ocaml is statically typed language.

    \item Pattern matching is the act of checking a given sequence of tokens for the presence of the constituents of some pattern. 
          It is like ``switch case'' structure.

          \begin{lstlisting}
              match n with
              | 1 -> "VE477"
              | 2 -> "VE475"
              | 3 -> "VE482";;
          \end{lstlisting}
    
    \item Pattern matching binds each case with a unique operation, thus if the operation raises some type error, 
          the compiler will catch it and raise warnings. In this way, type safety is increased.

\end{enumerate}

\section{Interview problems}

This part is done in the lab.


\end{document}